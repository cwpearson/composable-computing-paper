\documentclass[11pt]{article}

\usepackage{sectsty}
\usepackage{graphicx}
\usepackage{xcolor}

\newcommand{\brian}[1]{{\color{green}{Brian: #1}}} % Brian
\newcommand{\carl}[1]{{\color{red}{Carl: #1}}} % carl
\newcommand{\christian}[1]{{\color{purple}{Christian: #1}}} % Christian
\newcommand{\ihsin}[1]{{\color{blue}{I-Hsin: #1}}} % I-Hsin
\newcommand{\kaoutar}[1]{{\color{brown}{Kaoutar: #1}}} % Kaoutar
\newcommand{\wenmei}[1]{{\color{orange}{Wen-Mei: #1}}} % wen-mei


% Margins
\topmargin=-0.45in
\evensidemargin=0in
\oddsidemargin=0in
\textwidth=6.5in
\textheight=9.0in
\headsep=0.25in

\title{ Composable Computing: Opportunities and Challenges }
\author{
     Brian Pan\\
     H3 Systems
\and
    Carl Pearson\\
    Sandia National Labs
\and
    Christian Pinto\\
    IBM Research Europe
\and
    I-Hsin Chung\\
    IBM T. J. Watson Research
\and
    Kaoutar Maghraoui\\
    IBM T. J. Watson Research
\and
    Wen-Mei Hwu\\
    Nvidia
}
\date{Draft of \today}

\begin{document}
\maketitle	

An example citation~\cite{gao2016network}

\brian{a note from Brian}

\carl{a note from Carl}

\christian{a note from Christian}

\kaoutar{a note from Kaoutar}

\ihsin{a note from I-Hsin}

\wenmei{a note from Wen-Mei}

\section{What is Composable Computing}

\carl{A system in which the hardware resources are abstract, and can be provisioned to produce a reified computer, especially through a programmatic management interface.}

\section{The Current State of Composable Computing}

\section{Advantages and Disadvantages of Composable Computing}

Advantages vs. Cloud
\begin{itemize}
\item \carl{cloud-like operations on sensitive data in-house}
\item \carl{don't write your competitors a check every month}
\end{itemize}

Advantages vs. Traditional Datacenter
\begin{itemize}
\item \carl{improved utilization of resources through bin-packing over workloads; compute-to-memory utilization varies by 3 orders of magnitude~\cite{gao2016network}}
\item \carl{Flexible for workloads that change over time}
\end{itemize}

Advantages vs. time-shared computing
\begin{itemize}
\item \carl{larger pools of addressable memory may remove parallelization overheads; i.e. sometimes we parallelize because a single node is too small}
\item \carl{Remote memory can be backed by SSD as long as latency requirements are met}
\end{itemize}

Disadvantages vs. Cloud
\begin{itemize}
\item \carl{datacenter capital costs}
\end{itemize}

Disadvantages vs. Traditional datacenter
\begin{itemize}
\item \carl{resource orchestration}
\end{itemize}

Disadvantages vs. time-shared computing
\begin{itemize}
\item \carl{power use / energy efficiency}
\end{itemize}

\section{Combining HPC Workflows with Composable Systems}

\carl{Easy access to various hardware configurations}

\carl{Batch submission systems}

\carl{Programming models: hardware, hypervisor, and application level}

\carl{Finer-grained resource accounting}

\section{Robustness}
\carl{proactive approaches: self-healing and pre-emptive migration are natural when everything is virtualized anyway}

\carl{reactive: checkpoint and restart (storage is virtualized), migration (PEs are virtualized), replication (ask for 1 GPU, get 3)}

\section{Changes to the Datacenter}

\carl{hot-plug?}

\carl{drivers?}

\section{Coping with Heterogeneous Devices}

\carl{profiling discussion?}

\section{Looking Toward the Future}

\subsection{Challenges}

\carl{Latency}

\carl{Security \& resource isolation}

\carl{power / energy consumption}

\subsection{Opportunities}



\bibliographystyle{plain} % We choose the "plain" reference style
\bibliography{main} % Entries are in the main.bib file

\end{document}
